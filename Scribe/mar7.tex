% This is the LaTeX template file for homework for CSC549, Advanced Algorithm Design and Analysis.  When preparing homework solutions for this class, please use this template.
%
% This LaTeX template was created by the UC Berkeley EECS dept.
%
%

\documentclass[twoside]{article}
\setlength{\oddsidemargin}{0.25 in}
\setlength{\evensidemargin}{-0.25 in}
\setlength{\topmargin}{-0.6 in}
\setlength{\textwidth}{6.5 in}
\setlength{\textheight}{8.5 in}
\setlength{\headsep}{0.75 in}
\setlength{\parindent}{0 in}
\setlength{\parskip}{0.1 in}

%
% add packages if you need them here:
%

\usepackage{amsmath,amsfonts,graphicx}
\usepackage{algorithm,caption}
\usepackage[noend]{algpseudocode}
%
% the following macro is used to generate the header.
%
\newcommand{\homework}[4]{
   \pagestyle{myheadings}
   \thispagestyle{plain}
   \newpage
   \setcounter{page}{1}
   \noindent
   \begin{center}
   \framebox{
      \vbox{\vspace{2mm}
    \hbox to 6.28in { {\bf CSC 349-07: Design and Analysis of Algorithms
		\hfill Winter 2023} }
       \vspace{4mm}
       \hbox to 6.28in { {\Large \hfill P vs NP:  March 7, 2023 \hfill} }
       \vspace{2mm}
       \hbox to 6.28in { {\it Instructor: Rodrigo Canaan \hfill Scribe: Ishaan Sathaye} }
      \vspace{2mm}}
   }
   \end{center}
}


% use these for theorems, lemmas, proofs, etc.
\newtheorem{theorem}{Theorem}
\newtheorem{lemma}[theorem]{Lemma}
\newtheorem{proposition}[theorem]{Proposition}
\newtheorem{claim}[theorem]{Claim}
\newtheorem{corollary}[theorem]{Corollary}
\newtheorem{definition}[theorem]{Definition}
\newenvironment{proof}{{\bf Proof:}}{\hfill\rule{2mm}{2mm}}

% **** you may define additional macros here:
\algnewcommand\algorithmicinput{\textbf{Input:}}
\algnewcommand\INPUT{\item[\algorithmicinput]}

\algnewcommand\algorithmicoutput{\textbf{Output:}}
\algnewcommand\OUTPUT{\item[\algorithmicoutput]}

\begin{document}
\homework


% **** your solution goes here:

% Some general latex examples and examples making use of the macros follow.  

\section{Announcements}
{\em Make sure to write all announcements here.}
\begin{itemize}
    \item Dr. Migler reminded everyone to be healthy and kind and to sign up for a scribe day.
    \item Second bullet point 
    \item Third bullet point 
\end{itemize}

\section{Searching} 

\subsection{Problem Statement}

\noindent\fbox{
\parbox{\textwidth}{
{\bf Search}

    {\bf Input:} A list of numbers, sorted in increasing order,
    $A = [a_0,a_1,a_2,a_3,\dots a_{n-1}]$ and a number $x$.

    {\bf Goal:} Return a position of $x$ in $A$ if $x\in A$ or return a statement that $x$ is not in the list.
    
}
}

\subsection{Pseudocode for {\sc BinarySearch} }
\begin{algorithm}[H]
\caption*{{\sc BinarySearch($A$, $x$, min, max)}}\label{alg:binary_search}
\begin{algorithmic}[1]

\INPUT An array of $n$ sorted numbers $A$, number $x$, integers min,
and max. 
(Initially $min = 0$ and $max = n-1$)

\OUTPUT A position of $x$ in $A$ if $x\in A$, or $-1$ if $x\notin A$.

\If {max < min}

\Return $-1$


\Else


\If {$x<A[\lfloor (max + min)/2 \rfloor ]$}


\Return BinarySearch($A, x$, min,$\lfloor (max + min)/2 \rfloor  -1$) 


\ElsIf {$x>A[\lfloor (max + min)/2 \rfloor ]$}


\Return BinarySearch($A, x$ ,$\lfloor (max + min)/2 \rfloor  +1, max$) 


\Else
 

\Return $\lfloor (max + min)/2 \rfloor $
\EndIf


\EndIf

\end{algorithmic}
\end{algorithm} 

\subsection{Correctness of {\sc BinarySearch}}
Notice that in the output statement for {\sc BinarySearch} we said
that {\bf A} position of $x$ is returned because we don't know which
position of $x$ will be returned (if $x$ occurs multiple times). For
example if we call  {\sc BinarySearch($\{2, 2, 3, 5, 5, 7\}$, $5$, min,
  max)} then the second occurrence of 5 is returned. 

Also notice that for {\sc BinarySearch($A$, $x$, min, max)} we aren't
``throwing away'' half of the list at every iteration. Rather, we are
reducing the size of the interval under consideration ($(min,max)$).

For the base case: if $max < min$ then the interval that we are
considering is empty, thus $x$ couldn't possibly be in that interval.

Again, we must prove that {\sc BinarySearch($A$, $x$, min, max)} is correct.
We will prove the following lemmas:
\begin{lemma}\label{proof:binary_search_firsttry}
  If $x\in A$, then {\sc BinarySearch($A$, $x$, min, max)} returns a
  position of $x$ in $A$.
\end{lemma}

\begin{lemma}\label{proof:binary_search_not_in}
  If $x\notin A$, then {\sc BinarySearch($A$, $x$, min, max)} returns $-1$.
\end{lemma}

We will prove Lemma~\ref{proof:binary_search_not_in} by contradiction:

\begin{proof}
  Suppose $x\notin A$, further suppose for a contradiction that {\sc BinarySearch($A$,
    $x$, min, max)} returns something other than $-1$. Note that the
  only time that {\sc BinarySearch($A$,
    $x$, min, max)} returns something other than $-1$ is in line 5 of
  the algorithm. The only way to enter line 5 of the algorithm is if
  $x = A[\lfloor (max + min)/2\rfloor]$. However, if $x\notin A$ then
  this will never occur. We have reached our contradiction. Of course
  we should also point out that our algorithm terminates... {\em Why?}
\end{proof}

Now for the proof of Lemma~\ref{proof:binary_search_firsttry}. Divide
and conquer is a recursive technique and therefore induction is
usually the proof technique for the proofs of correctness for these
types of algorithms. We will restate our lemma to make it more
suitable for a proof by induction. Note that we are going to use
induction over the size of the interval that we are considering
($(min,max)$).

\begin{lemma}
  If $x\in A(min,max)$ and $max - min = n$, then {\sc
    BinarySearch($A$, $x$, min, max)} returns a position of $x$ in
  $A$.
\end{lemma}

\begin{proof}
We proceed by induction of the size of the interval under
consideration.

  {\bf Base Case:} Suppose that $x\in A(min,max)$ and $max - min =
  0$. Then $min = max = \lfloor (max + min)/2\rfloor$. Also, since
  $x\in A(min,max)$, it must be that
  $x = A[min] = A[max] = A[\lfloor (max + min)/2\rfloor]$. So line 5
of {\sc BinarySearch($A$, $x$, min, max)} returns
$min = max = \lfloor (max + min)/2\rfloor$.

{\bf (Strong) Inductive Hypothesis: }If $x\in A(min,max)$ and
$max - min \leq k$, then {\sc BinarySearch($A$, $x$, min, max)}
returns a position of $x$ in $A$.

{\bf Inductive Step: }Suppose $x\in A(min,max)$ and $max - min = k +
1$. There are three possibilities:
\begin{enumerate}
\item $x < A[\lfloor (max + min)/2\rfloor]$
\item $x > A[\lfloor (max + min)/2\rfloor]$
\item $x = A[\lfloor (max + min)/2\rfloor]$
\end{enumerate}

Case 1: Because the list is sorted and $x$ is in the interval 
$A[(min,max)]$, it must be that $x$ is in the first half of the
interval, $A[(min, \lfloor (max + min)/2\rfloor - 1)]$.

What does {\sc BinarySearch($A$, $x$, min, max)} do in this case? 

In this case, in line 3, {\sc BinarySearch($A$, $x$, min, max)}
returns {\sc BinarySearch($A$, $x$, min, $\lfloor (max + min)/2\rfloor
  - 1)$}.

Note that $x \in A[(min, \lfloor (max + min)/2\rfloor - 1)]$ and
$(\lfloor (max + min)/2\rfloor - 1) - min \leq k$, so our inductive
hypothesis is met and {\sc BinarySearch($A$, $x$, min, max)}
returns a position of $x$ in $A$.

Case 2: Symmetric to Case 1: because the list is sorted and $x$ is
in the interval from $A[(min,max)]$, it must be that $x$ is in the
second half of the interval,
$A[( \lfloor (max + min)/2\rfloor + 1, max)]$.

What does {\sc BinarySearch($A$, $x$, min, max)} do in this case? 

In this case, in line 4, {\sc BinarySearch($A$, $x$, min, max)}
returns {\sc BinarySearch($A$, $x$, $\lfloor (max + min)/2\rfloor
  + 1, max)$}.

Note that $x \in A[(\lfloor (max + min)/2\rfloor + 1, max)]$ and
$max - (\lfloor (max + min)/2\rfloor + 1) \leq k$, so our inductive
hypothesis is met and {\sc BinarySearch($A$, $x$, min, max)}
returns a position of $x$ in $A$.

Case 3: If $x = A[\lfloor (max + min)/2\rfloor]$, then line 5 of {\sc
  BinarySearch($A$, $x$, min, max)} correctly returns $\lfloor (max +
min)/2\rfloor$.

Therefore by the principle of mathematical induction, we have the result.

\end{proof}

Notice that we used {\em strong induction} in the above proof. This is
because the size of the problem is being reduced by about $\frac{n}{2}$
at each stage, so having information about a problem of size $n-1$
(which you would get from weak or regular induction) would not be
useful.

We have proven that {\sc BinarySearch($A$, $x$, min, max)} is
correct. The next step is to evaluate the running time. 


\end{document}